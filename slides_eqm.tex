% Created 2021-03-11 Thu 19:52
% Intended LaTeX compiler: pdflatex
\documentclass[notheorems, bigger]{beamer}
\usepackage[utf8]{inputenc}
\usepackage[T1]{fontenc}
\usepackage{graphicx}
\usepackage{grffile}
\usepackage{longtable}
\usepackage{wrapfig}
\usepackage{rotating}
\usepackage[normalem]{ulem}
\usepackage{amsmath}
\usepackage{textcomp}
\usepackage{amssymb}
\usepackage{capt-of}
\usepackage{hyperref}
%
% American mathematical society libs
%

% imports
\usepackage{amsmath,amssymb,latexsym,amsfonts,amsthm, mathtools}

% definitions
\newtheorem{theorem}{Theorem}[section]
\newtheorem{lemma}{Lemma}[section]
\newtheorem{proposition}{Proposition}[theorem]
\newtheorem{corollary}{Corollary}[theorem]

\theoremstyle{definition}
\newtheorem{definition}{Definition}[section]
\newtheorem{example}{Example}[section]
\newtheorem{remark}{Remark}[section]

%
% Commands and aliases
%

% Complex numbers set
\newcommand{\C}{\mathbb{C}}

% Real numbers set
\newcommand{\R}{\mathbb{R}}

% Rational numbers set
\newcommand{\Q}{\mathbb{Q}}

% Integer numbers set
\newcommand{\Z}{\mathbb{Z}}

% Natural numbers set
\newcommand{\N}{\mathbb{N}}

% Prime numbers set
\renewcommand{\P}{\mathbb{P}}

% Ring of integers set
\renewcommand{\O}{\mathcal{O}}
\newcommand{\Ok}{\mathcal{O}_K}

% Ring of integers set
\newcommand{\Id}{\mathfrak{I}}

% Canonical basis
\newcommand{\Cb}{\mathcal{C}}


% Definition
\newcommand{\defsym}{\vcentcolon =}

% Id Est i.e.
\newcommand{\ie}{\textit{i.e.}}

% Maximum real subfield
\newcommand{\maxrs}{\Q(\zeta_{p} + \zeta_{p} ^{-1})}

% K_\R inner space
\newcommand{\krspace}{(K_\R,\langle{\cdot,\cdot}\rangle_\tau)}
\usetheme{metropolis}
\author{Candidate: Laura Viglioni \\ Supervisor: Prof. Dr. Ricardo Dahab}
\date{March 12, 2021}
\title{A study of some practical impacts of twisted embeddings in lattice-based cryptography}
\hypersetup{
 pdfauthor={Candidate: Laura Viglioni \\ Supervisor: Prof. Dr. Ricardo Dahab},
 pdftitle={A study of some practical impacts of twisted embeddings in lattice-based cryptography},
 pdfkeywords={},
 pdfsubject={},
 pdfcreator={Emacs 27.1 (Org mode 9.4.4)}, 
 pdflang={English}}
\begin{document}

\maketitle

\section{Basic definitions}
\label{sec:orge83d328}
\begin{frame}[label={sec:orgd7ae2d4}]{Lattices}
A \textbf{lattice} \(\Lambda \subset \R^n\) is a subgroup of the additive group \(\R^n\).
\end{frame}
\begin{frame}[label={sec:org26b5cc7}]{Lattices}
\begin{text}
  In other words, given $m$ linear independent vectors in $\R^n$, the set
  $\{v_1, v_2, ..., v_m\}$ is called a \textbf{basis} for $\Lambda$ and the lattice may be defined
  by:

  \begin{equation*}
    \Lambda := \left\{x = \sum_{i=1}^m{\lambda_iv_i} \in \R^n \; | \; \lambda_i \in \Z\right\}.
  \end{equation*}

  That is, any $\lambda \in \Lambda$ can be written as $\lambda = Mv$, where $M$ is the
  \textbf{generator matrix} of $\Lambda$ where each row is a vector from the basis and
  $v \in \Z^n$.
\end{text}
\end{frame}
\begin{frame}[label={sec:orga6649a6}]{Lattices and cryptography}
In the last two decades, lattice-based cryptosystems have become an important field in the cryptography community, since these cryptosystems rely on mathematical problems we believe are hard and quantum-resistant, such as the Shortest Vector Problem and the Shortest Independent Vectors Problem.
\end{frame}
\begin{frame}[label={sec:org7e5f115}]{Lattices problems}
\begin{block}{Gap Shortest Vector Problem}
\begin{text}
  For an approximation factor $\gamma  = \gamma(n) \geq 1$, the $GapSVP_\gamma $ is: given a lattice
  $\Lambda$ and length $d > 0$, output \textbf{YES} if $\lambda_1(\Lambda) \leq d$ and \textbf{NO} if
  $\lambda_1(L) > \gamma d$. 
\end{text}
\end{block}
\begin{block}{Shortest Independent Vectors Problem}
\begin{text}
  For an approximation factor $\gamma = \gamma(n) \geq 1$, the $SIVP_\gamma$ is: given a lattice $\Lambda$, output $n$ linearly independent lattice vectors of length at most $\gamma(n) \cdot \lambda_n(\Lambda)$.
\end{text}
\end{block}
\end{frame}
\begin{frame}[label={sec:org4307949}]{The \emph{H} space}
\begin{text}
  Let $r,s,n \in \Z_+$ such that $n = r + 2s > 0$. The space $H \subset \C^n$ is defined
  as:
  \begin{equation*}
    H = \{(a_1,\dots, a_r, b_1,\dots, b_s, \overline{b_1}, \dots, \overline{b_s}) \in \C^n\},
  \end{equation*}

  where $a_i \in \R, \; \forall i \in \{1,\dots,r\}$ and $b_j \in \C, \; \forall \; j \in \{1,\dots,
  s\}$.
\end{text}
\end{frame}
\begin{frame}[label={sec:org64998c6}]{The \emph{H} space}
\begin{text}
  For all $x = \left(x_1, \dots, x_n\right), y = \left(y_1, \dots, y_n\right) \in H$ the space
  $H$ is endowed with inner product $\langle {x,y} \rangle_H$ defined as:
  \begin{equation*}
    \langle {x,y} \rangle_H = \sum_{i=1}^n{x_i \overline{y_i}} = \sum_{i=1}^r{x_i y_i} + \sum_{i=1}^s{x_{i+r} \overline{y_{i+r}}} + \sum_{i=1}^s{\overline{x_{i+r}} y_{i+r}}.
  \end{equation*}

  The $\ell_2$-norm and infinity norm of any $x \in H$ are defined as $\|x\| =
  \sqrt{\langle{x,x}\rangle_H}$ and $\|x\|_\infty = \max{\{ |x_i| \}}_{i=1}^n $.
\end{text}
\end{frame}
\begin{frame}[label={sec:org0b699d3}]{Number Fields}
\begin{text}
  For $K, L$ two fields, we denote by $L/K$ a \textbf{field extension} if  $K \subseteq
  L$. Then $L$ is said to be an \textbf{ extension field} over $K$, or just an
  \textbf{extension} over $K$. In a field extension $L/K$, $L$ has the structure of a vector space over $K$.


  A field extension is called a  \textbf{number field} when it is over the rational field $\Q$. 
\end{text}
\end{frame}
\begin{frame}[label={sec:org956cfa8}]{Twisted embeddings}
\begin{text}
  Let $K$ and $L$ be two field extensions and a homomorphism $\phi: K \rightarrow L$. $\phi$ is
  said to be a \textbf{$\Q$-homomorphism} if $\phi(a) = a, ; \forall a \in \Q$.
  \\


  A $\Q$-homomorphism $\phi: K \rightarrow \C$ is called an \textbf{embedding}.
\end{text}
\end{frame}
\begin{frame}[label={sec:orgac1f6a1}]{Twisted embeddings}
\begin{theorem}
  If $K$ is a number field with degree $n$ then there are
  exactly $n$ embeddings $\sigma_i : K \rightarrow \C$ where by $\sigma_i(\theta) =
  \theta_i$ where $\theta_i \in \C$ is a distinct zero of $K$'s
  minimum polynomial.
\end{theorem}
\end{frame}

\begin{frame}[label={sec:org5c35171}]{Twisted embeddings}
\begin{text}
  The homomorphism $\sigma: K \rightarrow \R^r \times \C^s$, where $(r,s)$ is the signature of $K$, is the \textbf{canonical embedding} and is defined by:
  \[
  \sigma(x) = \left(\sigma_1(x), \ldots , \sigma_r(x), \sigma_{r+1}(x), \ldots, \sigma_{r+s}(x) \right).
\]

  Note that we could rewrite the canonical embedding as $\sigma : K \rightarrow \R^n,$

  \begin{align*}
    \sigma(x) = (& \sigma_1(x), \ldots , \sigma_r(x), \\
            & \Re(\sigma_{r+1}(x)), \Im(\sigma_{r+1}(x)), \ldots, \Re(\sigma_{r+s}(x)), \Im(\sigma_{r+s}(x)) ).
  \end{align*}

\end{text}
\end{frame}
\begin{frame}[label={sec:org7742a47}]{Algebraic lattices}
Let \(\{\omega_1,...,\omega_n\}\) be an integral basis of \(K\). The \(n\) vectors \(v_i = \sigma(\omega_i)
\in \R^n\) are linearly independent, so they define a full rank algebraic lattice
\(\Lambda = \Lambda(\Ok) = \sigma(\Ok)\).

The generator matrix of \(\Lambda = \sigma(\Ok)\) is defined by

\begin{equation*}
  \label{definition:gen-matrix-alg-lattices}
  \begin{pmatrix}
    \sigma_1(\omega_1) & $\dots$ &  \sigma_{r+2s}(\omega_1) \\
    & \vdots & \\
    \sigma_1(\omega_n) & $\dots$ & \sigma_{r+2s}(\omega_n) \\
  \end{pmatrix}.  
\end{equation*}
\end{frame}
\begin{frame}[label={sec:org008af64}]{Twisted embeddings and number fields}
\begin{text}
  An embedding creates the correspondence between a point $\lambda \in \Lambda \subset \R^n$ of an algebraic lattice.
  \begin{align*} 
    \lambda &= (\lambda_1,\dots,\lambda_{r+2s}) \in \Lambda \\
            &= \left( \sum_{i=1}^n{z_i\sigma_1(\omega_i)} , \dots , \sum_{i=1}^n{z_i\sigma_{r+2s}(\omega_i)} \right) \\
            &= \left( \sigma_1\left(   \sum_{i=1}^n{z_i\omega_i} \right) , \dots , \sigma_{r+2s} \left( \sum_{i=1}^n{z_i\omega_i}  \right) \right), 
  \end{align*}

  where $z_i \in \Z$. Since any element $x \in \Ok$ has the form $x =
  \sum_{i=1}^n{\lambda_i\omega_i}$, we can conclude that

  \begin{equation*}
    \lambda = \left( \sigma_1(x), \dots, \sigma_{r+2s}(x) \right) = \sigma(x).
  \end{equation*}
\end{text}
\end{frame}


\section{Learning problems}
\label{sec:org48c00ec}
\begin{frame}[label={sec:orgbce80ae}]{Learning from Parity}
\begin{text}
  Given $m$ vectors uniformly chosen  $a_i \gets \Z^n_2$ and some $\epsilon \in [0,1]$, we
  define the problem \textbf{Learning from Parity (LFP)} as:

  Find $s \in \Z^n_2$ such that, for $i \in \{1,\dots,m\}$
  $$ \langle{s, a_i}\rangle \; \approx_\epsilon \; b_i \;\; (mod\; 2). $$

  In other words, the equality holds with probability $1 - \epsilon$.
\end{text}
\end{frame}
\begin{frame}[label={sec:org0f7a595}]{Learning with Errors}
\begin{text}
Learning with Errors (LWE) is a generalization of LFP  with two new parameters
$p \in \P$ and $\chi$ a probability distribution on $\Z_p$ so that we have:
\[
  <s, a_i> \; \approx_\chi \; b_i \pmod p \;\;\; \text{or} \;\;\; <s, a_i> + \; e_i =  b_i \pmod p ,
\]
where $a_i \gets \Z^n_p$ uniformly and $e_i \gets \Z$ according to $\chi$.
\end{text}
\end{frame}
\begin{frame}[label={sec:orgd6b0cf0}]{Ring-LWE search}
\begin{text}
  Let $K$ be a number field, $R = \Ok$ its ring of integers and $R^\vee$ the
  codifferent ideal of $K$. Also let $K_\R$ be the tensor product $K \otimes_\Q \R$.


  Let $\Psi$ be a family of distributions over $K_\R$. The \textbf{search version of the $ring-LWE$ problem}, denoted $R-LWE_{q,\Psi}$, is defined as follows: given access to arbitrarily many independent samples from $A_{s,\psi}$ for some arbitrary $s \in R_q^\vee$ and $\psi \in \Psi$, find $s$.
\end{text}
\end{frame}
\begin{frame}[label={sec:org50124b4}]{Twisted Ring-LWE}
\begin{text}
  For a totally positive element $\tau \in F$, let $\psi_\tau$ denote an error distribution
  over the inner product $\langle{\cdot,\cdot}\rangle_\tau$ and $s \in R^\vee_q$ (the “secret”) be an
  uniformly randomized element. The \emph{Twisted Ring-LWE distribution}
  $\mathcal{A}_{s,\psi_\tau}$ produces samples of the form
  \[
    a, b = a \cdot s + e \pmod{qR^\vee} \in R_q \times K_\R/qR^\vee.
\]
\end{text}
\end{frame}
\begin{frame}[label={sec:org90f0492}]{Twisted Ring-LWE hardness}
Solving the Twisted Ring-LWE is as hard as solving the usual Ring-LWE.
\begin{theorem}
  \label{theorem:twisted-ring-lwe-hardness}
  Let $K$ be an arbitrary number field, and let $\tau \in F$ be totally positive.
  Also, let $(s,\psi)$ be randomly chosen from $(U(R_q^\vee)\times \Psi)$ in $(K_\R,\langle{\cdot,\cdot}\rangle_{\tau=1})$.
  Then there is a polynomial-time reduction from $\mbox{Ring-LWE}_{q,\psi}$ to $\mbox{Ring-LWE}^\tau_{q,\psi_\tau}$.
\end{theorem}
\end{frame}
\section{Twisted R-LWE cryptosystem}
\label{sec:org9f527a3}
\begin{frame}[label={sec:orga082b1e}]{Cryptosystem presented by Ortiz et al.}
\begin{itemize}
\item Let \(R\) be an \emph{m}-th cyclotomic ring and \(p, q \in \Z\) coprime numbers.
\item The message space is defined as \(R_p\).
\item Consider that \(\phi_\tau\) is an error distribution over \(\krspace\) and \(\lfloor{\cdot}\rceil\) denotes a valid discretization to (cosets) of \(R^\vee\) or \(pR^\vee\).
\item \(\hat{m} = m/2\) if \(m\) is even, otherwise \(\hat{m} = m\).
\item For any \(\overline{a} \in \Z_q\), let \([[\overline{a}]]\) denote the unique representative \(a \in (\overline{a} + q\Z) \cap [-q/2, q/2)\), which is entry-wise extended to polynomials.
\end{itemize}
\end{frame}
\begin{frame}[label={sec:org09fb4ff}]{Cryptosystem presented by Ortiz et al.}
 \begin{itemize}
\item \textbf{Key generation}: choose a uniformly random $a \in R_q$. Choose $x
  \longleftarrow \lfloor{\phi_\tau}\rceil$ and $e \longleftarrow \lfloor{p \cdot \phi_\tau}\rceil_{pR^\vee}$. Output $(a,b = \hat{m}\cdot(a \cdot x + e)
  \mod{qR} ) \in R_q \times R_q$ as the public key and $x$ as the secret key.
\item \textbf{Encryption}: choose $z \longleftarrow  \lfloor{\phi_\tau}\rceil_R^\vee$, $e' \longleftarrow \lfloor{p \cdot
    \phi_\tau}\rceil_{pR^\vee}$ and  $e'' \longleftarrow \lfloor{p \cdot \phi_\tau}\rceil_{t^{-1}\mu +pR^\vee}$, where $\mu \in R_p$ is
  the word to be encrypted. Let $u = \hat{m} \cdot (a \cdot z + e') \mod{qR}$ and $v =
  z \cdot b + e'' \in R_q^\vee$. Output $(u,v) \in R_q \times R^\vee_q$.
\item \textbf{Decryption}: Given the encrypted message $(u,v)$, compute $v - u
  \cdot x \mod{qR^\vee}$, and decode it to $d = [[v - u \cdot x]] \in R^\vee$. Output $\mu = t \cdot
  d \bmod{pR}$. 
\end{itemize}
\end{frame}
\section{Objectives}
\label{sec:org4397ce7}
\begin{frame}[label={sec:org099c7cc}]{Main goal}
\begin{itemize}
\item Validate the idea of using twisted embeddings in cryptography
\item Explore the theoretical and the practical aspects of this proposal
\end{itemize}
\end{frame}
\begin{frame}[label={sec:org0d4f2de}]{Practical aspects}
\begin{itemize}
\item Compare implementations and instances of the Twisted Ring-LWE and Ring-LWE
\item Maximum real subfield versus the cyclotomic power-of-two
\item Search for proper sizes of keys and messages
\end{itemize}
\end{frame}
\begin{frame}[label={sec:org4b50e57}]{Theoretical aspects}
\begin{itemize}
\item Study the polynomial arithmetic of the maximal real subfield
\item Study the relation between the orthonormal basis and the efficient conversion between lattice points and elements of number field
\item Examine if it is possible to achieve a satisfactory efficiency with non-orthonormal basis
\end{itemize}
\end{frame}
\section{Methodology and timeline}
\label{sec:orge02fe9b}
\begin{frame}[label={sec:orgc0adc97}]{Methodology}
\begin{itemize}
\item \textbf{Literature Review:} review proposals of new cryptosystems, such as \emph{NTTRU}.
\item \textbf{Theoretical experiments:} perform experiments using algebra
  libraries to discover twist factors and to discover orthonormal bases.
\item \textbf{Experimental outcome:} to calculate the expansion factor of the polynomial \(f(x)\) that defines the ring \(\Z[x]/f(x)\). Adapt or develop algorithms for polynomial multiplication.
\item \textbf{Implementation:} implement a Twisted Ring-LWE based cryptosystem.
\item \textbf{Practical experiments:} to estimate the cost in terms of clock cycles, also key and message sizes.
\end{itemize}
\end{frame}
\begin{frame}[label={sec:org3088abd}]{Timeline}
\begin{itemize}
\item First and second semesters of 2021
\begin{itemize}
\item Study the Twisted Ring LWE problem and implementation.
\item Perform theoretical experiments with number fields, twist factors and lattices.
\item Calculate the expansion factor and adapt/develop algorithms for polynomial multiplication.
\end{itemize}
\item First and second semesters of 2022
\begin{itemize}
\item Implement a Twisted Ring-LWE based cryptosystem.
\item Compare instances of Ring LWE and Twisted Ring LWE, \ie, analyze the cryptosystem in both terms of clock cycles and key sizes.
\item Defense of dissertation.
\end{itemize}
\end{itemize}
\end{frame}
\section{Thank you!}
\label{sec:org6487b7d}
\end{document}